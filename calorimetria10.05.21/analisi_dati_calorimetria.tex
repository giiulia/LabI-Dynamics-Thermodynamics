\documentclass[a4paper]{article}
\usepackage[utf8]{inputenc}




\title{Calorimetria \\ analisi dati}
\author{Ali Matteo,\\Broggi Diana, Cantarini Giulia}
\date{ }

\usepackage{tabularx}
\usepackage{natbib}
%\usepackage[demo]{graphicx}
\usepackage{graphicx}

%\usepackage[margin=1.0in]{geometry}

\usepackage{tikz}

\usepackage{subcaption}
\usepackage{caption}
\usepackage{amsmath, amsthm}

\usepackage{mathrsfs}

\usepackage{pgfplots}

\usepackage{caption}



%\pgfplotsset{width=4cm,compat=1.9}

\theoremstyle{definition}
\newtheorem{rich}{richiamo matematico}[section]





%roba che crea comando per centrare immagine dentro immagine piu grande
%https://tex.stackexchange.com/a/308286
\newlength{\imagew}
\newlength{\imageh}
\newlength{\legendw}
\newlength{\legendh}
\newlength{\legendx}
\newlength{\legendy}
\newcommand{\graphicswithlegend}[6]{
	\setlength{\imagew}{#1}
	\settoheight{\imageh}{\includegraphics[width=\imagew]{#2}}
	
	\setlength{\legendw}{#3\imagew}
	\settoheight{\legendh}{\includegraphics[width=\legendw]{#4}}
	
	\setlength{\legendx}{\imagew}
	\addtolength{\legendx}{-\legendw}
	\addtolength{\legendx}{-#5\imagew}
	
	\setlength{\legendy}{\imageh}
	\addtolength{\legendy}{-\legendh}
	\addtolength{\legendy}{-#6\imageh}
	
	\includegraphics[width=\imagew]{#2}%
	\llap{
		\hspace{-\the\legendx}
		\raisebox{\legendy}{\includegraphics[width=\legendw]{#4}}
		\hspace{\the\legendx}
	}
}



\begin{document}
	\pagenumbering{arabic}
	\maketitle
\subsubsection*{strumenti}
calorimetro delle mescolanze di Regnault\\
bilancia con sensibilità pari a 0.01g\\
termometro con sensibilità di 0.05 \(^{\circ}C\)\\
alimentatore a corrente continua con sensibilità 1 Volt\\
cronometro con sensibilità 1 sec\\\\
	
\makebox[\textwidth]{
	{	
		\normalsize
		Stima della massa equivalente:	\(m_{e} = m_{2}\frac{(T_{2}-T_{eq})}{(T_{eq}-T_{1})} - m_{1}\)
	}
}


\begin{figure}[!htbp]
	\captionsetup{labelformat=empty}

	\makebox[1 \textwidth][c]{ 
		\begin{tabular}{c|cccc}
			\hline
			\hline
			  &I misura & II misura & III misura & IV misura \\
			\hline
			\(m_{1}\) (Kg)&0.09972 & 0.09075 & 0.09975 & 0.14294\\
			\(m_{2}\) (Kg)&0.10343 & 0.09253 &0.10040 & 0.08663\\
			\(T_{1}\) (\(^{\circ}C\))& 24.50 & 23.66 &23.64 &22.00  \\
			\(T_{2}\) (\(^{\circ}C\))& 85.50 & 40.30 & 53.20 & 88.00\\
			\(T_{eq}\) (\(^{\circ}C\))& 50.00 & 32.10 & 39.30 & 44.50\\
			\hline
			\hline
		\end{tabular}
	} %close centering
	\caption{per l'incertezza su tali misure abbiamo considerato la sensibilità dello strumento utilizzato}
\end{figure}	

\makebox[\textwidth]{
	{
		\normalsize
		\(\sigma_{m_{e}} = \sqrt{\left ( \frac{\partial m_{e}}{\partial m_{1}} \right )^{2}\sigma_{m_{1}}^{2} + \left ( \frac{\partial m_{e}}{\partial m_{2}} \right )^{2} \sigma_{m_{2}}^{2}+\left ( \frac{\partial m_{e}}{\partial T_{1}} \right )^{2} \sigma_{T_{1}}^{2}+\left ( \frac{\partial m_{e}}{\partial T_{2}} \right )^{2} \sigma_{T_{2}}^{2} + \left ( \frac{\partial m_{e}}{\partial T_{eq}} \right )^{2} \sigma_{T_{eq}}^{2}}\)
	}
}


\begin{figure}[!htbp]
	\captionsetup{labelformat=empty}
	
	\makebox[1 \textwidth][c]{ 
		\begin{tabular}{c|cccc}
			\hline
			\hline
			&I misura & II misura & III misura & IV misura \\
			\hline
			\(m_{e}\) (Kg) &\(0.0443 \pm 0.0006\) &\(-0.0009 \pm 0.0013\)& \(-0.0106 \pm 0.0007\)&  \(0.0245 \pm 0.0007\)\\
			\hline
			\hline
		\end{tabular}
	} %close centering
\end{figure}

calcolo della media pesata: \(\bar{m_{e}} = \frac{\sum m_{i}w_{i}}{\sum w_{i}} \pm \frac{1}{\sqrt{\sum w_{i}}} =  (0.0213 \pm 0.0004) Kg\)\\\\\\\\\\\\\\\\\\\\\\

\subsubsection*{determinazione del calore specifico del rame}

\makebox[\textwidth]{
	{	
		\normalsize
		Calcolo del calore specifico:	\(c_{s} = \frac{c_{acqua} (T_{eq}-T_{1})(m_{1}+m_{e})}{(T_{s}-T_{eq})m_{s}}\)
	}
}


\begin{figure}[!htbp]
	\captionsetup{labelformat=empty}
	\makebox[1 \textwidth][c]{ 
		\begin{tabular}{c|ccc}
			\hline
			\hline
			&I misura & II misura & III misura\\
			\hline
			\(m_{1}\) (Kg)&\(0.190740 \pm0.000008\) & \(0.22346  \pm 0.00001\)&  \(0.24877 \pm 0.00001\)\\
			\(m_{s}\) (Kg)&\(0.129920 \pm 0.000014\) & \( 0.129920 \pm 0.000014 \) & \( 0.129920 \pm 0.000014\) \\
			\(T_{1}\) (\(^{\circ}C\))& \(22.50\pm 0.05\) & \(22.30 \pm 0.05\) & \(22.50 \pm 0.05\) \\
			\(T_{2}\) (\(^{\circ}C\))& \(100.00 \pm 0.05\) & \(100.00 \pm 0.05\) & \(100.00 \pm 0.05\)\\
			\(T_{eq}\) (\(^{\circ}C\))& \(26.00 \pm 0.05\) & \(25.50 \pm 0.05\) & \(25.40 \pm 0.05\)\\
			\hline
			\hline
		\end{tabular}
	} %close centering

\end{figure}	

\makebox[\textwidth]{
	{

		\normalsize
		\(\sigma_{c_{s}} = \sqrt{\left ( \frac{\partial c_{s}}{\partial m_{1}} \right )^{2}\sigma_{m_{1}}^{2} + \left ( \frac{\partial c_{s}}{\partial m_{s}} \right )^{2} \sigma_{m_{s}}^{2}+\left ( \frac{\partial c_{s}}{\partial T_{1}} \right )^{2} \sigma_{T_{1}}^{2}+\left ( \frac{\partial c_{s}}{\partial T_{s}} \right )^{2} \sigma_{T_{s}}^{2} + \left ( \frac{\partial c_{s}}{\partial T_{eq}} \right )^{2} \sigma_{T_{eq}}^{2}}\)
	}
}


\begin{figure}[!htbp]
	\captionsetup{labelformat=empty}
	\makebox[1 \textwidth][c]{ 
		\begin{tabular}{c|ccc}
			\hline
			\hline
			&I misura & II misura & III misura \\
			\hline
			\(c_{s}\) (J/Kg\(^{\circ}C\)) &\(322.7 \pm4.7\) &\(338.3 \pm 5.3\)& \(337.8 \pm  5.8\)\\
			\hline
			\hline
		\end{tabular}
	} %close centering
\end{figure}

calcolo della media pesata: \(\bar{c_{s}} = \frac{\sum cs_{i}w_{i}}{\sum w_{i}} \pm \frac{1}{\sqrt{\sum w_{i}}} =  (331.7 \pm 3.0)\)J/Kg\(^{\circ}C\)\\

\noindent Il valore atteso per il calore specifico del rame è 385 J/Kg\(^{\circ}C\), test di accuratezza: \(t = \frac{\left | c_{s atteso} - c_{s osservato} \right |}{\sigma_{cs}} = 18\) \(\rightarrow\) la probabilità che la discrepanza con il valore atteso sia dovuta solo ad errori casuali è inferiore al 0.3 \(\%\).

\subsubsection*{verifica del valore della costante di Joule} \makebox[\textwidth]{
	{	
		\normalsize
		Calcolo della costante di Joule:	\(J = \frac{IV\Delta t}{c_{acqua}(m_{1}+m_{e})}\)
	}
}


\begin{figure}[!htbp]
	\captionsetup{labelformat=empty}
	\caption{\(T_{iniziale}\) = 25 \(^{\circ}C\)}
	\makebox[1 \textwidth][c]{ 
		\begin{tabular}{r|cccc|c}
			\hline
			\hline
			&I misura & II misura & III misura & IV misura & totale \\
			\hline
			\(T_{finale}\) (\(^{\circ}C\))&27.5 & 30.5 & 33 & 36.5 &\\
			\(c_{acqua}\) (J/Kg\(^{\circ}C\))& 4180.0 & 4178.8 &4178.3 &4178.3 &4180.0\\
			\(\Delta T\) (\(^{\circ}C\)) & 2.5 & 3.0 & 2.5 & 3.5 & 11.5\\
			\(\Delta t\) (s)&120 & 120 &120 &120 & 480\\
			\hline
			\hline
		\end{tabular}
	} %close centering
	\caption{costanti: \(m_{1}\)=\((0.30998 \pm 0.00001)\) Kg, \(I\)=\( (2.2 \pm 0.1) \) A, 	\(V\)=\((15 \pm 1)\) Volt}
\end{figure}	

\makebox[\textwidth]{
	{
		\normalsize
		\(\sigma_{J} = \sqrt{\left ( \frac{\partial J}{\partial \Delta T} \right )^{2}\sigma_{\Delta T}^{2} + \left ( \frac{\partial J}{\partial \Delta t} \right )^{2} \sigma_{\Delta t}^{2}+\left ( \frac{\partial J}{\partial m_{1}} \right )^{2} \sigma_{m_{1}}^{2}+\left ( \frac{\partial J}{\partial I} \right )^{2} \sigma_{I}^{2} + \left ( \frac{\partial J}{\partial V} \right )^{2} \sigma_{V}^{2}}\)
	}
}


\begin{figure}[!htbp]
	\captionsetup{labelformat=empty}
	\makebox[1 \textwidth][c]{ 
		\begin{tabular}{c|cccc|c}
			\hline
			\hline
			&I misura & II misura & III misura & IV misura & totale\\
			\hline
			J &\(1.1 \pm 0.1\) &\(0.95 \pm  0.08\)& \( 1.1 \pm 0.1\)&  \( 0.82 \pm  0.07\) & \(0.99 \pm 0.08\)\\
			\hline
			\hline
		\end{tabular}
	} %close centering
\end{figure}

calcolo della media pesata: \(\bar{J} = \frac{\sum m_{i}w_{i}}{\sum w_{i}} \pm \frac{1}{\sqrt{\sum w_{i}}} = (0.97 \pm  0.04) Kg\)\\\\
\makebox[\textwidth]{
	{
		\normalsize
conversione da J/J in J/Cal: J = \((4076.7 \pm 159.8)\) J/Cal
	}
}
.\\\\
Confronto con il valore atteso di 4.186 J/cal: \(t = \frac{\left | J_{atteso} - J_{osservato}\right |}{\sigma_{J}}= 0.68\) \(\rightarrow\) l'accuratezza di tale misura è pari al \(74.2 \%\).
	
\subsubsection*{misura del calore latente del ghiaccio} \makebox[\textwidth]{
	{	
		\normalsize
		Calcolo del calore latente del ghiaccio:	\(\lambda = \frac{(m_{1}+m_{e})c_{acqua(26^{\circ}C)}(T_{1}-T_{eq})- m_{2}c_{ghiaccio}(T_{0}-T_{2}) - m_{2}(T_{eq}-T_{0})c_{acqua(0^{\circ}C)}}{m_{2}}\)
	}
}


\begin{figure}[!htbp]
	\captionsetup{labelformat=empty}
	\makebox[1 \textwidth][c]{ 
		\begin{tabular}{c|c}
			\hline
			\hline
			\(m_{1}\) & \(0.24098\) Kg\\
			\(m_{2}\) & \(0.01420\) Kg\\
			\(T_{1}\) & \(26.00\) \(^{\circ}C\) \\
			\(T_{2}\) & \(-17.00\) \(^{\circ}C\)\\
			\(T_{eq}\) & \(19.50\) \(^{\circ}C\)\\
			\(c_{acqua(26^{\circ}C)}\)& 4179 J/Kg\(^{\circ}C\)\\
			\(c_{acqua(0^{\circ}C)}\) &  4217.7 J/Kg\(^{\circ}C\)\\
			\hline
			\hline
		\end{tabular}
	} %close centering
	\caption{l'incertezza su tali misure è pari alla sensibilità dello strumento utilizzato}
\end{figure}	

\makebox[\textwidth]{
	{
		\normalsize
		\(\sigma_{\lambda} = \sqrt{\left ( \frac{\partial \lambda}{\partial m_{1}} \right )^{2}\sigma_{m_{1}}^{2} + \left ( \frac{\partial \lambda}{\partial m_{2}} \right )^{2} \sigma_{m_{2}}^{2}+\left ( \frac{\partial \lambda}{\partial T_{1}} \right )^{2} \sigma_{t_{1}}^{2}+\left ( \frac{\partial \lambda}{\partial T_{2}} \right )^{2} \sigma_{T_{2}}^{2} + \left ( \frac{\partial \lambda}{\partial T_{eq}} \right )^{2} \sigma_{T_{eq}}^{2}}\)
	}
}
.\\\\
\makebox[\textwidth]{
	{
		Risultato: \(\lambda =  384678.2 \pm 1383.1 = (3.847 \pm 0.014) \cdot 10^{5} \) J/Kg
	}
}
.\\
Confronto con il valore atteso di 333 \( \cdot 10^{3}\) J/Kg: \(t = \frac{\left | \lambda_{atteso} - \lambda_{osservato}\right |}{\sigma_{\lambda}} = 37\) \(\rightarrow\) la probabilità che la discrepanza sia dovuta ad errori casuali è inferiore al \(0.3 \%\).
\end{document}